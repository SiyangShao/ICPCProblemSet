\chapter{解题方法}

\section{题目类型}

\subsection{最优解构造}

\subsection{存在解构造}

\subsection{k-th 解构造}

\section{通用构造思想}

\subsection{抽屉原理}

\subsection{随机构造}

\subsection{递归构造}

\subsection{分类构造}

\section{经典构造方法}

\subsection{图构造}

\begin{enumerate}
    \item 构造若干个强连通分量,这些强连通分量之间形成一条链
    \item 构造环
    \item 利用二分图匹配或者欧拉回路进行无向图边定向
    \item 利用 dfs 树,进行边匹配
\end{enumerate}


\ptitle{[CF1763E] Node Pairs}{zmy}
\prob 要求构造一个有向图,恰好有 $p (1 \leq p \leq 2 \cdot 10^5)$ 个点对互相可达。在这个前提下,要求使用的顶点数尽可能少。顶点数相同时,要求单向可达的点对尽可能多
\sol 每个有向图都可以被缩点成一个 DAG,对于每个强连通分量内部肯定是两两可达的,且夸强连通分量的点对一定不是双向可达的。因此我们可以想到用 DP 来计算需要的最小顶点数。再次之上,为了使得单向可达的点对尽可能多,我们的 DAG 一定是一条链的形状,因此我们可以同时 dp 单向可达的点对个数。

\ptitle{[ARC161D] Everywhere is Sparser than Whole (Construction)}{zmy}
\prob 给定节点数 $N$ 和图的密度 $D$。定义密度为整个图的边数除以点数的结果。现在要求构造一个图,使得整个图的密度为 $D$,且不存在一个子集的密度大于或者等于 $D$
\sol 我们考虑建立几个圈,当 $D=1$ 时,对于每个点 $i$ 我们连一条 $(i, i+1)$。当 $D=2$ 时,在此基础上,对于每个点 $i$,我们多连一条 $(i, i+2)$,以此类推。

\ptitle{[ARC161F] Everywhere is Sparser than Whole (Judge)}{zmy}
\prob 给定一个图保证整个图的密度恰好为 $D$。现在问你能否找到一个真子集的引导子图,使得其密度大于等于 $D$。
\sol 假设这个子集为 $S$,那么其补集 $S^{\prime}$ 连接的边的数量应该小于 $D|S^{\prime}|$。换句话说,如果不存在子集 $S$ 的话,原图的所有子集 $S^{\prime}$ 的链接的边数都得大于等于 $D|S^{\prime}|$。这里我们可以使用 Hall 定理。考虑构建二部图,左边是所有的边,右边是所有的点。左边每个点可以使用 1 次, 右边每个使用 $D$ 次。用 dinic 跑最大流,如果满流,则意味着存在完美匹配,根据 Hall 定理可知前面描述的条件能够被满足。但即使满足了,还有一种情况是,存在一个真子集其密度为 $D$。一个引理是,我们按照前面完美匹配的方案,对原图进行定向。原图存在一个真子集使其密度等于 $D$,当且仅当定向后的有向图存在多个强连通分量。我们再跑一遍 tarjan 即可。


\ptitle{[UCUP Stage 0J] Perfect Matching}{zmy}
\prob 给定一张包含 $n$ 个顶点的无向图($n$ 是偶数)以及 $n$ 个整数 $a_1, a_2,\cdots, a_n$。对于任意满足 $1 \leq i < j \leq n$ 的正整数 $i$ 和 $j$,若 $|i − j| = |a_i − a_j|$($|x|$ 表示 $x$ 的绝对值)则在无向图中顶点 $i$ 和顶点 $j$ 之间连一条无向边。求一个完美匹配,或表明不存在完美匹配。$2 \leq n \leq 10^5$,保证所有测试数据中 $n$ 之和不超过 $10^6$.
\sol 两个点可以相连,当且仅当 $a_i+i=a_j+j$ 或者 $a_i-i=a_j-j$。我们将所有可能的 $a_i+i$ 和 $a_i-i$ 建立点,$i$ 则视为链接这两个点的一条边。因此问题被转化成我们要将所有的边划分成若干条长度为 $2$ 的链。我们首先用并查集求出所有的连通分量。当且仅当某个连通分量里面边的数量为奇数时无解。否则,我们对这个连通分量进行 dfs,自底向上贪心匹配。如果当前剩余连向孩子的边数为偶数,两两配对即可。若为奇数,将这条边和当前节点和它父亲连的那条边进行配对即可。   

\subsection{序列构造}
