% !TEX TS-program = xelatex
% !TEX encoding = UTF-8 Unicode
% !Mode:: "TeX:UTF-8"

\documentclass[a4paper]{report}
\usepackage{cite}
\usepackage{xeCJK}
\usepackage{multicol}
\usepackage{framed}
\usepackage{tcolorbox}
\usepackage{amssymb}
\usepackage{amsmath}
\usepackage{amsthm}
\usepackage[colorlinks,linkcolor=blue]{hyperref}
\usepackage{indentfirst} 
\usepackage{geometry}
\usepackage{longtable}
\usepackage{xspace}
\usepackage{listings}
\setlength{\parindent}{2em}
\usepackage{amsmath}
\usepackage{algorithm}  
\usepackage{algorithmicx}  
\usepackage{algpseudocode}

\addtolength{\topmargin}{-54pt}
\setlength{\oddsidemargin}{0.33cm}  % 3.17cm - 1 inch
\setlength{\evensidemargin}{\oddsidemargin}
\setlength{\textwidth}{14.66cm}
\setlength{\textheight}{24.00cm}    % 24.62

\newtheorem{definition}{定义}[section]
\newtheorem{theorem}{定理}[section]
\newtheorem{lemma}{引理}[section]
\newtheorem{cproof}{证明}[section]
\newtheorem{notation}{记号}[section]
\newtheorem{property}{性质}[section]
\newtheorem{corollary}{推论 }[section] 

\newenvironment{problem}{
\newcommand\title[2]{\noindent\textbf{##1}  [\emph{##2}]}
\newcommand\discription[1]{\textbf{题目描述} {##1}}
\newcommand\keywords[1]{\textbf{题目类型} {##1}}
\newcommand\solution[1]{\textbf{题目分析} {##1}}}

\newenvironment{exprob}{
\scriptsize
}


\newcommand{\ptitle}[2]{\noindent\textbf{#1}  [\emph{#2}] \par}
\newcommand{\prob}{\textbf{题目描述}\;}
\newcommand{\sol}{\par \textbf{题目分析}\;}
\newcommand{\mypara}[1]{\paragraph{#1}\mbox{}\\}

%%% 自定义宏
\def\indent{\hspace{20pt}}
\geometry{a4paper,left=2cm,right=2cm,top=1.5cm,bottom=1.5cm}

\begin{document}

% \begin{CJK*}{UTF8}{gbsn}

\title{ICPC Problem Set}
\author{NTRLover}
\date{2023.12.11}

\maketitle

\chapter{解题报告}
\begin{center}
\begin{longtable}{l c c c c c }

Date & Problem ID & Title & Difficulty & Type & Person   \\ 
\hline
2022-12-20 & CF1763E & Node Pairs & 2200 & 图论、最优性构造 & zmy  \\
2023-12-11 & CF1904E & Tree Queries & unkown & 图论, 树论 & shaosy \\
2023-12-14 & CF1904F & Beautiful Tree & unkown & 图论, 树论 & shaosy \\
2023-12-14 & CF1903F & Babysitting & 2500 & 图论 & shaosy \\
2023-12-22 & CF1913E & Matrix Problem & unkown & 图论 & shaosy \\
2023-12-22 & CF1905E & One-X & unkown & 分治, 线段树 & shaosy \\
\hline

\label{tab:practice_index}
\end{longtable}
\end{center}
\section{Codeforces}

\ptitle{[CF1763E] Node Pairs}{zmy}
\prob 要求构造一个有向图,恰好有 $p (1 \leq p \leq 2 \cdot 10^5)$ 个点对互相可达。在这个前提下,要求使用的顶点数尽可能少。顶点数相同时,要求单向可达的点对尽可能多
\sol 每个有向图都可以被缩点成一个 DAG,对于每个强连通分量内部肯定是两两可达的,且夸强连通分量的点对一定不是双向可达的。因此我们可以想到用 DP 来计算需要的最小顶点数。再次之上,为了使得单向可达的点对尽可能多,我们的 DAG 一定是一条链的形状,因此我们可以同时 dp 单向可达的点对个数。

\ptitle{[CF1904E] Tree Queries}{shaosy}
\prob 给定一棵树, 多组询问. 每组询问指定一个询问点和一些删除点, 问将删除点删除后树中询问点可达的最远距离为多少
\sol 离线处理. 首先, 询问某一点在书中可达最远距离等价于将该点视为树的根并询问树的深度. 考虑如何换根时更新树的深度. 考虑dfs序. 通过维护每个点的in和out时刻, 配合线段树, 可以用换根dp类似的思想来动态维护每个点当前的深度. 根通过边u-v从u换到v等价于, 给v的子树深度区间减1, 给其他节点深度加1. 然后讨论删除点. 删除点如果在原树中不是新的根节点(x) 的父节点,则可以视作将删除点的子树的深度区间减去正无穷. 而如果是, 则可以视作将除了删除点与点x相连的子树之外所有点的深度区间减去正无穷. 在这种更新下, 当前树的深度就是所有点深度的最大值.

\ptitle{[CF1904F] Beautiful Tree}{shaosy}
\prob 给定一棵树和多组限制条件,要求给树上每一个点赋值使得其成为一个permutation并满足所有限制条件. 每组限制条件形如 u-v-x, 要求使得在路径u-v上的点中x的值最大/最小
\sol 考虑树链剖分+线段树优化建图. 这样, 每一段树上路径可以用不超过$\log{n}$个线段, 每个线段又不超过$\log{n}$个点表示. 建立两颗线段树, 分别表示区间最大的点/区间最小的虚拟点, 而后暴力连边. 暴力连边后跑一边拓扑排序即可. 若存在环,则说明不存在可行解. 线段树优化建图中需要注意为了避免出现自环, 对路径u-v向路径x连边/或其反边连边时, 需要特判x是否在该路径中. 如果在该路径中必须裂成u-x和x-v两段(不包括x)
\section{POI}

\ptitle{Step Traversing a Tree}{pufanyi}

\prob 给你一棵 $n\ (n\le 5000)$ 个点的树,构造一个长度为 $n$ 的 $permutation$,使得 $\forall i, \mathrm{dist}\left\{p_i, p_{i+1}\right\}\le 3$。
\sol $f_u$ 从 $u$ 开始,到 $u$ 的一个叶子节点结束,遍历以 $u$ 为根子的答案树;$g_u$ 表示从 $u$ 的一个叶子节点开始,到 $u$ 结束遍历以 $u$ 为根子树的答案。发现两者可以相互递归调用。
\chapter{解题方法}

\section{题目类型}

\subsection{最优解构造}

\subsection{存在解构造}

\subsection{k-th 解构造}

\section{通用构造思想}

\subsection{抽屉原理}

\subsection{随机构造}

\subsection{递归构造}

\subsection{分类构造}

\section{经典构造方法}

\subsection{图构造}

\begin{enumerate}
    \item 构造若干个强连通分量,这些强连通分量之间形成一条链
    \item 构造环
    \item 利用二分图匹配或者欧拉回路进行无向图边定向
    \item 利用 dfs 树,进行边匹配
\end{enumerate}


\ptitle{[CF1763E] Node Pairs}{zmy}
\prob 要求构造一个有向图,恰好有 $p (1 \leq p \leq 2 \cdot 10^5)$ 个点对互相可达。在这个前提下,要求使用的顶点数尽可能少。顶点数相同时,要求单向可达的点对尽可能多
\sol 每个有向图都可以被缩点成一个 DAG,对于每个强连通分量内部肯定是两两可达的,且夸强连通分量的点对一定不是双向可达的。因此我们可以想到用 DP 来计算需要的最小顶点数。再次之上,为了使得单向可达的点对尽可能多,我们的 DAG 一定是一条链的形状,因此我们可以同时 dp 单向可达的点对个数。

\ptitle{[ARC161D] Everywhere is Sparser than Whole (Construction)}{zmy}
\prob 给定节点数 $N$ 和图的密度 $D$。定义密度为整个图的边数除以点数的结果。现在要求构造一个图,使得整个图的密度为 $D$,且不存在一个子集的密度大于或者等于 $D$
\sol 我们考虑建立几个圈,当 $D=1$ 时,对于每个点 $i$ 我们连一条 $(i, i+1)$。当 $D=2$ 时,在此基础上,对于每个点 $i$,我们多连一条 $(i, i+2)$,以此类推。

\ptitle{[ARC161F] Everywhere is Sparser than Whole (Judge)}{zmy}
\prob 给定一个图保证整个图的密度恰好为 $D$。现在问你能否找到一个真子集的引导子图,使得其密度大于等于 $D$。
\sol 假设这个子集为 $S$,那么其补集 $S^{\prime}$ 连接的边的数量应该小于 $D|S^{\prime}|$。换句话说,如果不存在子集 $S$ 的话,原图的所有子集 $S^{\prime}$ 的链接的边数都得大于等于 $D|S^{\prime}|$。这里我们可以使用 Hall 定理。考虑构建二部图,左边是所有的边,右边是所有的点。左边每个点可以使用 1 次, 右边每个使用 $D$ 次。用 dinic 跑最大流,如果满流,则意味着存在完美匹配,根据 Hall 定理可知前面描述的条件能够被满足。但即使满足了,还有一种情况是,存在一个真子集其密度为 $D$。一个引理是,我们按照前面完美匹配的方案,对原图进行定向。原图存在一个真子集使其密度等于 $D$,当且仅当定向后的有向图存在多个强连通分量。我们再跑一遍 tarjan 即可。


\ptitle{[UCUP Stage 0J] Perfect Matching}{zmy}
\prob 给定一张包含 $n$ 个顶点的无向图($n$ 是偶数)以及 $n$ 个整数 $a_1, a_2,\cdots, a_n$。对于任意满足 $1 \leq i < j \leq n$ 的正整数 $i$ 和 $j$,若 $|i − j| = |a_i − a_j|$($|x|$ 表示 $x$ 的绝对值)则在无向图中顶点 $i$ 和顶点 $j$ 之间连一条无向边。求一个完美匹配,或表明不存在完美匹配。$2 \leq n \leq 10^5$,保证所有测试数据中 $n$ 之和不超过 $10^6$.
\sol 两个点可以相连,当且仅当 $a_i+i=a_j+j$ 或者 $a_i-i=a_j-j$。我们将所有可能的 $a_i+i$ 和 $a_i-i$ 建立点,$i$ 则视为链接这两个点的一条边。因此问题被转化成我们要将所有的边划分成若干条长度为 $2$ 的链。我们首先用并查集求出所有的连通分量。当且仅当某个连通分量里面边的数量为奇数时无解。否则,我们对这个连通分量进行 dfs,自底向上贪心匹配。如果当前剩余连向孩子的边数为偶数,两两配对即可。若为奇数,将这条边和当前节点和它父亲连的那条边进行配对即可。   

\subsection{序列构造}


% \end{CJK*}
\end{document}


